\chapter{Introduction to Symmetry}

\section{Intuition and Motivation}

The idea of symmetry is the the object has a property that remains invariant under a transformation. For example, if we rotate a square by 90 degrees, the square remains the same. However, symmetry is more than a geometric concept. It is a fundamental concept in mathematics and physics.

\begin{example}[Polygons]
    We can rotate the following triangle with respect to $O$ by $120^{\circ}$, and the triangle remains the same. This triangle has rotational symmetry. 

    \begin{center}
        \begin{tikzpicture}[baseline=(current bounding box.center)]
            % A regular triangle with a dot in the center
            \draw (0, 0) -- (2, 0) -- (1, 1.732) -- cycle;

            % The center of the triangle
            \node[graph-node, label=above:$O$] at (1, 0.577) {};

            % The labels of the vertices
            \node[below] at (0, 0) {$B$};
            \node[below] at (2, 0) {$C$};
            \node[above] at (1, 1.732) {$A$};
        \end{tikzpicture}%
        \hfil$\xrightarrow{\text{rotate by } 120^{\circ}}$\hfil
        \begin{tikzpicture}[baseline=(current bounding box.center)]
            % A regular triangle with a dot in the center
            \draw (0, 0) -- (2, 0) -- (1, 1.732) -- cycle;

            % The center of the triangle
            \node[graph-node, label=above:$O$] at (1, 0.577) {};

            % The labels of the vertices
            \node[below] at (0, 0) {$A$};
            \node[below] at (2, 0) {$B$};
            \node[above] at (1, 1.732) {$C$};
        \end{tikzpicture}
    \end{center}

    Moreover, we can also reflect the triangle with respect to the line $l$ passing through $O$, and the triangle remains the same. This triangle has reflection symmetry.

    \begin{center}
        \begin{tikzpicture}[baseline=(current bounding box.center)]
            % A regular triangle with a dot in the center
            \draw (0, 0) -- (2, 0) -- (1, 1.732) -- cycle;

            % The center of the triangle
            \node[graph-node, label=above:$O$] at (1, 0.577) {};

            % The labels of the vertices
            \node[below] at (0, 0) {$B$};
            \node[below] at (2, 0) {$C$};
            \node[above] at (1, 1.732) {$A$};

            % The line l with the label
            \draw[dashed] (1, 2.5) -- (1, -0.5);
            \node[below] at (1, -0.5) {$l$};
        \end{tikzpicture}%
        \hfil$\xrightarrow{\text{reflect with respect to } l}$\hfil
        \begin{tikzpicture}[baseline=(current bounding box.center)]
            % A regular triangle with a dot in the center
            \draw (0, 0) -- (2, 0) -- (1, 1.732) -- cycle;

            % The center of the triangle
            \node[graph-node, label=above:$O$] at (1, 0.577) {};

            % The labels of the vertices
            \node[below] at (0, 0) {$C$};
            \node[below] at (2, 0) {$B$};
            \node[above] at (1, 1.732) {$A$};
        \end{tikzpicture}
    \end{center}

    Is there any other symmetry? Yes, we can combine the two symmetries above. We first rotate the triangle by $120^{\circ}$, and then reflect it with respect to $l$. This triangle has both rotational and reflection symmetry.

    % Graph of the triangle
    \begin{center}
        \begin{tikzpicture}[baseline=(current bounding box.center)]
            % A regular triangle with a dot in the center
            \draw (0, 0) -- (2, 0) -- (1, 1.732) -- cycle;

            % The center of the triangle
            \node[graph-node, label=above:$O$] at (1, 0.577) {};

            % The labels of the vertices
            \node[below] at (0, 0) {$B$};
            \node[below] at (2, 0) {$C$};
            \node[above] at (1, 1.732) {$A$};
        \end{tikzpicture}%
        \hfil$\xrightarrow{\text{rotate by } 120^{\circ}}$\hfil
        \begin{tikzpicture}[baseline=(current bounding box.center)]
            % A regular triangle with a dot in the center
            \draw (0, 0) -- (2, 0) -- (1, 1.732) -- cycle;

            % The center of the triangle
            \node[graph-node, label=above:$O$] at (1, 0.577) {};

            % The labels of the vertices
            \node[below] at (0, 0) {$A$};
            \node[below] at (2, 0) {$B$};
            \node[above] at (1, 1.732) {$C$};

            % The line l with the label
            \draw[dashed] (1, 2.5) -- (1, -0.5);
            \node[below] at (1, -0.5) {$l$};
        \end{tikzpicture}%
        \hfil$\xrightarrow{\text{reflect with respect to } l}$\hfil
        \begin{tikzpicture}[baseline=(current bounding box.center)]
            % A regular triangle with a dot in the center
            \draw (0, 0) -- (2, 0) -- (1, 1.732) -- cycle;

            % The center of the triangle
            \node[graph-node, label=above:$O$] at (1, 0.577) {};

            % The labels of the vertices
            \node[below] at (0, 0) {$B$};
            \node[below] at (2, 0) {$A$};
            \node[above] at (1, 1.732) {$C$};
        \end{tikzpicture}
    \end{center}
\end{example}

The above example is a very simple one. However, given an general object, it is not easy to find all its symmetries. We can label the vertices of the triangle with $A, B, C$, then permute the labels. 

\begin{center}
    \begin{tikzpicture}[baseline=(current bounding box.center)]
        % A regular triangle with a dot in the center
        \draw (0, 0) -- (2, 0) -- (1, 1.732) -- cycle;

        % The center of the triangle
        \node[graph-node, label=above:$O$] at (1, 0.577) {};

        % The labels of the vertices
        \node[draw,fill=white,circle] at (0, 0) {?};
        \node[draw,fill=white,circle] at (2, 0) {?};
        \node[draw,fill=white,circle] at (1, 1.732) {?};
    \end{tikzpicture}
\end{center}

Since the transformations are linear, they preserve linearity. This, it suffices to consider the transformations of the vertices. 

\begin{example}[Continued]
    The following table shows all the permutations of the vertices of the triangle. 

    \begin{table}[ht!]
        \centering
        \begin{tabular}{|c|c|c|c|}
            \hline
            Identity & $A$ & $B$ & $C$ \\
            \hline
            Rotation & $C$ & $A$ & $B$ \\
            \hline
            Reflection & $A$ & $C$ & $B$ \\
            \hline
            Rotation + Reflection & $C$ & $B$ & $A$ \\
            \hline \hline
            & $B$ & $A$ & $C$ \\
            \hline
            & $B$ & $C$ & $A$ \\
            \hline
        \end{tabular}
    \end{table}

    As we can see, there are six transformations of the vertices, each of which corresponds to a symmetry of the triangle. 
\end{example}

Naively, given an square, one would argue that there are $24$ ways to permute the vertices, and thus $24$ symmetries. However, this is not true. There are certain permutations that are not symmetries. 

\begin{center}
    \begin{tikzpicture}[baseline=(current bounding box.center)]
        % The square
        \draw (0, 0) -- (1, 0) -- (1, 1) -- (0, 1) -- cycle;

        % The labels of the vertices
        \node[below] at (0, 0) {$A$};
        \node[right] at (1, 0) {$B$};
        \node[above] at (1, 1) {$C$};
        \node[left] at (0, 1) {$D$};
    \end{tikzpicture}
    $\xrightarrow{\text{swap } A \text{ and } B}$
    \begin{tikzpicture}[baseline=(current bounding box.center)]
        % The square
        \draw (0, 0) -- (1, 0) -- (1, 1) -- (0, 1) -- cycle;

        % The labels of the vertices
        \node[below] at (0, 0) {$B$};
        \node[right] at (1, 0) {$A$};
        \node[above] at (1, 1) {$C$};
        \node[left] at (0, 1) {$D$};
    \end{tikzpicture}
\end{center}

\section{Symmetric Group}

\begin{definition}[Symmetric Group]\index{Symmetric Group}\label{def:symmetric_group}
    The \term{symmetric group}, denoted $S_n$, is the set of all permutations of $n$ elements $1, 2, \dots, n$.
\end{definition}

\begin{definition}[Identity Permutation]\index{Identity Permutation}\label{def:identity_permutation}
    The \term{identity permutation} is the permutation that does not change the order of the elements.
\end{definition}

\begin{example}
    The identity permutation of $S_3$ is the identity permutation of $1, 2, 3$.
\end{example}

\begin{definition}[Transposition]\index{Transposition}\label{def:transposition}
    A \term{transposition} is a permutation that swaps two elements and leaves the other elements unchanged.
\end{definition}

\begin{example}
    The following are some transpositions of $S_3$.
    \begin{itemize}
        \item $2, 1, 3$ swaps $1$ and $2$.
        \item $1, 3, 2$ swaps $2$ and $3$.
        \item $3, 2, 1$ swaps $1$ and $3$.
    \end{itemize}
\end{example}

\begin{definition}[Cycle]\index{Cycle}\label{def:cycle}
    A \term{cycle} is a permutation that moves the first element to the second, the second to the third, and so on, and the last element to the first.
\end{definition}

\begin{example}
    The cycle $3, 2, 1$ moves $1$ to $3$, $3$ to $2$, and $2$ to $1$.
\end{example}

\begin{definition}[Permutation]\index{Permutation}\label{def:permutation}
    A \text{permutation} is a way to order $n$ elements. We codify them in ``cycles''
\end{definition}

\begin{example}
    Consider $S_3$. 

    \begin{table}[ht!]
        \centering
        \begin{tabular}{c c c c c c}
            1 2 3 & 1 2 3 & 1 2 3 & 1 2 3 & 1 2 3 & 1 2 3 \\
            \hline 
            1 2 3 & 1 3 2 & 3 2 1 & 2 1 3 & 3 1 2 & 2 3 1 \\
            (1)(2)(3) & (1)(23) & (13)(2) & (12)(3) & (132) & (123) \\
        \end{tabular}
    \end{table}

    Here, $(1)(23)$ means

    \begin{itemize}
        \item $1$ goes to $1$.
        \item $2$ goes to $3$, and $3$ goes to $2$.
    \end{itemize}
\end{example}

\begin{example}
    Consider the following permutation. 

    \begin{table}[ht!]
        \centering
        \begin{tabular}{c|c}
            1 2 3 4 5 6 7 & 1 2 3 4 5 6 7 \\
            \hline
            3 4 2 1 7 5 6 & \color{blue}2 3 1 4 6 5 7 \\
            \color{blue}(1324)(576) & (1 2 3)(5 6)
        \end{tabular}
    \end{table}
\end{example}

\begin{example}
    Suppose you have two permutations $\sigma$ and $\tau$:

    \begin{listu}
        \item $\sigma = (1 2)(3 4 5 6)$
        \item $\tau = (1 6 5 4)(3 2)$
    \end{listu}

    What happens if we perform one after the other?

    \begin{listu}
        \item $\sigma$ first, $\tau$ second\footnote{Note that we read from right to left.}: $\color{red}(1 6 5 4)(3 2) \color{blue}(1 2)(3 4 5 6) \color{black} = (1 6 5 4)(3 2)(1 2)(3 4 5 6)$

        \begin{listu}
            \item We start with $1$: $1 \to 2 \to 3$, so $1 \to 3$. 
            \item We then consider $3$: $3 \to 4 \to 1$, so $3 \to 1$.
            \item Now, we consider $2$: $2 \to 1 \to 6$, so $2 \to 6$.
            \item $6 \to 3 \to 2$, so $6 \to 2$.
            \item $4 \to 5 \to 4$, so $4 \to 4$.
            \item $5 \to 6 \to 5$, so $5 \to 5$.
        \end{listu}

        Thus, we get \[(1 3)(2 6)(4)(5). \]

        \item $\tau$ first, $\sigma$ second: $\color{red}(1 2)(3 4 5 6) \color{blue}(1 6 5 4)(3 2) \color{black} = (1 2)(3 4 5 6)(1 6 5 4)(3 2)$

        \begin{listu}
            \item We start with $1$: $1 \to 6 \to 4$, so $1 \to 3$. 
            \item We then consider $3$: $4 \to 5 \to 1$, so $4 \to 1$.
            \item \dots
        \end{listu}

        Eventually, we get \[(1 3)(2 4)(5)(6). \]
    \end{listu}
        
    It is important to note that the order of the permutations matters.
\end{example}

The above example demonstrates an important property of permutations: closed under composition. That is, if we ``merge'' two permutations, we get another permutation.

\newpage
\begin{table}[ht!]
    \centering
    \renewcommand{\arraystretch}{1.25}
    \begin{tabular}{c|c|c|c|c|c|c}
        $\circ$ & $\mathds{1}$ & (12) & (13) & (23) & (123) & (132) \\
        \hline
        $\mathds{1}$ & & & & & \\
        \hline
        $(12)$ & & & & & \\
        \hline
        $(13)$ & & & & & \\
        \hline
        $(23)$ & (23) & (132) & (123) & $\mathbb{I}$ & (13) & (12)\\
        \hline
        $(123)$ & & & & & \\
        \hline
        $(132)$ & & & & &
    \end{tabular}
\end{table}

This is a multiplication table of $S_3$. Symmetries of the same group have the same multiplication table, despite the fact that they are different permutations.

\begin{remark}
    Note that in the above table of $S_3$, we have $(1 2 3) = (2 3)(1 3)$, and $(1 3 2) = (2 3)(1 2)$. \bred{All the permutations can be written as a composition of transpositions.} 

    It is important to note that this is not unique. For example, we can write $\mathds{1} = (12)(12)$. 
\end{remark}

\begin{theorem}
    The amount of transpositions needed to create a permutation preservers its parity.
\end{theorem}

In other words, if a permutation $\alpha$ can be expressed as a product of transpositions \[
    \alpha = \tau_1 \tau_2 \dots \tau_n \qquad \text{ and } \qquad \alpha = \sigma_1 \sigma_2 \dots \sigma_m
\] where $\tau$ and $\sigma$ are transpositions, then $n$ and $m$ have the same parity (both even or both odd). The smaller groups are called \term{alternating groups}.